\documentclass[12pt,a4paper]{article}
\usepackage[utf8]{inputenc}
\usepackage{polski}
\usepackage{graphicx}
\usepackage{hyperref}
\usepackage{geometry}
\geometry{margin=2.5cm}
\title{Scrum: Backlog sprintu\\\large Realizacja Projektu Informatycznego 2022/2023}
\author{Imię i nazwisko \\ Zespół projektowy: nazwa zespołu}
\date{Data oddania}

\begin{document}

\maketitle
\tableofcontents
\newpage

\section{O projekcie i produkcie}
\subsection{Opis projektu}
Tutaj wpisz krótki opis projektu oraz opisywanego produktu.

\subsection{Cel projektu}
Tutaj wpisz cel główny projektu.

\section{Oszacowanie rozmiaru backlogu produktu}
\subsection{Metoda szacowania}
Szacowanie wykonano metodą \textbf{Planning Poker}.

\subsection{Przebieg sesji Planning Poker}
Opisz krótko, jak przebiegała sesja Planning Poker (np. liczba uczestników, iteracje głosowania, ewentualne rozbieżności i ich rozwiązania).

\subsection{Wyniki szacowania}
\begin{itemize}
    \item User Story 1 – X story points
    \item User Story 2 – Y story points
    \item User Story 3 – Z story points
    \item itd.
\end{itemize}

(Wstaw tutaj opcjonalnie zdjęcia lub zrzuty ekranów.)

\section{Założenia i dobór zakresu sprintu}
\begin{itemize}
    \item \textbf{Pojemność zespołu}: np. 100 roboczogodzin
    \item \textbf{Rezerwa}: np. 20\% na spotkania Scrum, wsparcie
    \item \textbf{Średnia prędkość zespołu}: np. 25 story points na sprint
\end{itemize}

\subsection{Dobór elementów backlogu}
Opisz sposób doboru funkcji (np. priorytet biznesowy, zależności techniczne) i uzasadnij wybór.

\section{Cel sprintu}
Tutaj opisz najważniejszą wartość dodaną, która powinna być osiągnięta po sprincie.

\section{Backlog sprintu}
\subsection{Tabela Backlogu Sprintu}
\begin{tabular}{|p{4cm}|p{5cm}|p{3cm}|p{3cm}|}
\hline
\textbf{Element PB} & \textbf{Zadanie wytwórcze} & \textbf{Estymacja [godz]} & \textbf{Status}\\
\hline
User Story 1 & Zadanie 1.1, Zadanie 1.2 & 5h, 8h & To do\\
\hline
User Story 2 & Zadanie 2.1 & 6h & To do\\
\hline
User Story 3 & Zadanie 3.1, Zadanie 3.2 & 4h, 7h & To do\\
\hline
\end{tabular}

\section{Kryteria akceptacji}
\subsection{Przykładowe kryteria dla wybranych User Stories}
\begin{itemize}
    \item \textbf{User Story 1}:
    \begin{itemize}
        \item Funkcja działa zgodnie ze specyfikacją.
        \item Wszystkie przypadki testowe przechodzą.
        \item Dokumentacja jest uzupełniona.
    \end{itemize}
    \item \textbf{User Story 2}:
    \begin{itemize}
        \item Obsługuje błędne dane wejściowe.
        \item Interfejs użytkownika jest zgodny z wytycznymi UX.
    \end{itemize}
\end{itemize}

(Wstaw także zrzuty ekranu z narzędzia np. Jira, Trello itp.)

\section{Definicja ukończenia (Definition of Done)}
\begin{itemize}
    \item Kod znajduje się w repozytorium Git.
    \item Kod przeszedł code review.
    \item Wszystkie testy jednostkowe i integracyjne są zaliczone.
    \item Funkcjonalność jest przetestowana na środowisku testowym.
    \item Dokumentacja techniczna została zaktualizowana.
\end{itemize}

\end{document}
